% Options for packages loaded elsewhere
\PassOptionsToPackage{unicode}{hyperref}
\PassOptionsToPackage{hyphens}{url}
%
\documentclass[
]{article}
\usepackage{lmodern}
\usepackage{amsmath}
\usepackage{ifxetex,ifluatex}
\ifnum 0\ifxetex 1\fi\ifluatex 1\fi=0 % if pdftex
  \usepackage[T1]{fontenc}
  \usepackage[utf8]{inputenc}
  \usepackage{textcomp} % provide euro and other symbols
  \usepackage{amssymb}
\else % if luatex or xetex
  \usepackage{unicode-math}
  \defaultfontfeatures{Scale=MatchLowercase}
  \defaultfontfeatures[\rmfamily]{Ligatures=TeX,Scale=1}
\fi
% Use upquote if available, for straight quotes in verbatim environments
\IfFileExists{upquote.sty}{\usepackage{upquote}}{}
\IfFileExists{microtype.sty}{% use microtype if available
  \usepackage[]{microtype}
  \UseMicrotypeSet[protrusion]{basicmath} % disable protrusion for tt fonts
}{}
\makeatletter
\@ifundefined{KOMAClassName}{% if non-KOMA class
  \IfFileExists{parskip.sty}{%
    \usepackage{parskip}
  }{% else
    \setlength{\parindent}{0pt}
    \setlength{\parskip}{6pt plus 2pt minus 1pt}}
}{% if KOMA class
  \KOMAoptions{parskip=half}}
\makeatother
\usepackage{xcolor}
\IfFileExists{xurl.sty}{\usepackage{xurl}}{} % add URL line breaks if available
\IfFileExists{bookmark.sty}{\usepackage{bookmark}}{\usepackage{hyperref}}
\hypersetup{
  pdftitle={Análisis Preliminar},
  pdfauthor={Nathalie Bareño, Sebastián Rincón y Karen Vargas},
  hidelinks,
  pdfcreator={LaTeX via pandoc}}
\urlstyle{same} % disable monospaced font for URLs
\usepackage[margin=1in]{geometry}
\usepackage{color}
\usepackage{fancyvrb}
\newcommand{\VerbBar}{|}
\newcommand{\VERB}{\Verb[commandchars=\\\{\}]}
\DefineVerbatimEnvironment{Highlighting}{Verbatim}{commandchars=\\\{\}}
% Add ',fontsize=\small' for more characters per line
\usepackage{framed}
\definecolor{shadecolor}{RGB}{248,248,248}
\newenvironment{Shaded}{\begin{snugshade}}{\end{snugshade}}
\newcommand{\AlertTok}[1]{\textcolor[rgb]{0.94,0.16,0.16}{#1}}
\newcommand{\AnnotationTok}[1]{\textcolor[rgb]{0.56,0.35,0.01}{\textbf{\textit{#1}}}}
\newcommand{\AttributeTok}[1]{\textcolor[rgb]{0.77,0.63,0.00}{#1}}
\newcommand{\BaseNTok}[1]{\textcolor[rgb]{0.00,0.00,0.81}{#1}}
\newcommand{\BuiltInTok}[1]{#1}
\newcommand{\CharTok}[1]{\textcolor[rgb]{0.31,0.60,0.02}{#1}}
\newcommand{\CommentTok}[1]{\textcolor[rgb]{0.56,0.35,0.01}{\textit{#1}}}
\newcommand{\CommentVarTok}[1]{\textcolor[rgb]{0.56,0.35,0.01}{\textbf{\textit{#1}}}}
\newcommand{\ConstantTok}[1]{\textcolor[rgb]{0.00,0.00,0.00}{#1}}
\newcommand{\ControlFlowTok}[1]{\textcolor[rgb]{0.13,0.29,0.53}{\textbf{#1}}}
\newcommand{\DataTypeTok}[1]{\textcolor[rgb]{0.13,0.29,0.53}{#1}}
\newcommand{\DecValTok}[1]{\textcolor[rgb]{0.00,0.00,0.81}{#1}}
\newcommand{\DocumentationTok}[1]{\textcolor[rgb]{0.56,0.35,0.01}{\textbf{\textit{#1}}}}
\newcommand{\ErrorTok}[1]{\textcolor[rgb]{0.64,0.00,0.00}{\textbf{#1}}}
\newcommand{\ExtensionTok}[1]{#1}
\newcommand{\FloatTok}[1]{\textcolor[rgb]{0.00,0.00,0.81}{#1}}
\newcommand{\FunctionTok}[1]{\textcolor[rgb]{0.00,0.00,0.00}{#1}}
\newcommand{\ImportTok}[1]{#1}
\newcommand{\InformationTok}[1]{\textcolor[rgb]{0.56,0.35,0.01}{\textbf{\textit{#1}}}}
\newcommand{\KeywordTok}[1]{\textcolor[rgb]{0.13,0.29,0.53}{\textbf{#1}}}
\newcommand{\NormalTok}[1]{#1}
\newcommand{\OperatorTok}[1]{\textcolor[rgb]{0.81,0.36,0.00}{\textbf{#1}}}
\newcommand{\OtherTok}[1]{\textcolor[rgb]{0.56,0.35,0.01}{#1}}
\newcommand{\PreprocessorTok}[1]{\textcolor[rgb]{0.56,0.35,0.01}{\textit{#1}}}
\newcommand{\RegionMarkerTok}[1]{#1}
\newcommand{\SpecialCharTok}[1]{\textcolor[rgb]{0.00,0.00,0.00}{#1}}
\newcommand{\SpecialStringTok}[1]{\textcolor[rgb]{0.31,0.60,0.02}{#1}}
\newcommand{\StringTok}[1]{\textcolor[rgb]{0.31,0.60,0.02}{#1}}
\newcommand{\VariableTok}[1]{\textcolor[rgb]{0.00,0.00,0.00}{#1}}
\newcommand{\VerbatimStringTok}[1]{\textcolor[rgb]{0.31,0.60,0.02}{#1}}
\newcommand{\WarningTok}[1]{\textcolor[rgb]{0.56,0.35,0.01}{\textbf{\textit{#1}}}}
\usepackage{graphicx}
\makeatletter
\def\maxwidth{\ifdim\Gin@nat@width>\linewidth\linewidth\else\Gin@nat@width\fi}
\def\maxheight{\ifdim\Gin@nat@height>\textheight\textheight\else\Gin@nat@height\fi}
\makeatother
% Scale images if necessary, so that they will not overflow the page
% margins by default, and it is still possible to overwrite the defaults
% using explicit options in \includegraphics[width, height, ...]{}
\setkeys{Gin}{width=\maxwidth,height=\maxheight,keepaspectratio}
% Set default figure placement to htbp
\makeatletter
\def\fps@figure{htbp}
\makeatother
\setlength{\emergencystretch}{3em} % prevent overfull lines
\providecommand{\tightlist}{%
  \setlength{\itemsep}{0pt}\setlength{\parskip}{0pt}}
\setcounter{secnumdepth}{-\maxdimen} % remove section numbering
\ifluatex
  \usepackage{selnolig}  % disable illegal ligatures
\fi

\title{Análisis Preliminar}
\author{Nathalie Bareño, Sebastián Rincón y Karen Vargas}
\date{24 de mayo}

\begin{document}
\maketitle

\begin{Shaded}
\begin{Highlighting}[]
\FunctionTok{library}\NormalTok{(tidyverse)}
\end{Highlighting}
\end{Shaded}

\begin{verbatim}
## -- Attaching packages --------------------------------------- tidyverse 1.3.0 --
\end{verbatim}

\begin{verbatim}
## v ggplot2 3.3.3     v purrr   0.3.4
## v tibble  3.0.5     v dplyr   1.0.3
## v tidyr   1.1.2     v stringr 1.4.0
## v readr   1.4.0     v forcats 0.5.1
\end{verbatim}

\begin{verbatim}
## -- Conflicts ------------------------------------------ tidyverse_conflicts() --
## x dplyr::filter() masks stats::filter()
## x dplyr::lag()    masks stats::lag()
\end{verbatim}

\begin{Shaded}
\begin{Highlighting}[]
\FunctionTok{library}\NormalTok{(readxl)}
\FunctionTok{library}\NormalTok{(scales)}
\end{Highlighting}
\end{Shaded}

\begin{verbatim}
## 
## Attaching package: 'scales'
\end{verbatim}

\begin{verbatim}
## The following object is masked from 'package:purrr':
## 
##     discard
\end{verbatim}

\begin{verbatim}
## The following object is masked from 'package:readr':
## 
##     col_factor
\end{verbatim}

\begin{Shaded}
\begin{Highlighting}[]
\FunctionTok{library}\NormalTok{(ggpubr)}
\FunctionTok{library}\NormalTok{(stargazer)}
\end{Highlighting}
\end{Shaded}

\begin{verbatim}
## 
## Please cite as:
\end{verbatim}

\begin{verbatim}
##  Hlavac, Marek (2018). stargazer: Well-Formatted Regression and Summary Statistics Tables.
\end{verbatim}

\begin{verbatim}
##  R package version 5.2.2. https://CRAN.R-project.org/package=stargazer
\end{verbatim}

\begin{Shaded}
\begin{Highlighting}[]
\FunctionTok{library}\NormalTok{(GGally)}
\end{Highlighting}
\end{Shaded}

\begin{verbatim}
## Registered S3 method overwritten by 'GGally':
##   method from   
##   +.gg   ggplot2
\end{verbatim}

\begin{Shaded}
\begin{Highlighting}[]
\FunctionTok{library}\NormalTok{(ggthemes)}
\FunctionTok{library}\NormalTok{(tseries)}
\end{Highlighting}
\end{Shaded}

\begin{verbatim}
## Registered S3 method overwritten by 'quantmod':
##   method            from
##   as.zoo.data.frame zoo
\end{verbatim}

\begin{Shaded}
\begin{Highlighting}[]
\FunctionTok{library}\NormalTok{(lmtest)}
\end{Highlighting}
\end{Shaded}

\begin{verbatim}
## Loading required package: zoo
\end{verbatim}

\begin{verbatim}
## 
## Attaching package: 'zoo'
\end{verbatim}

\begin{verbatim}
## The following objects are masked from 'package:base':
## 
##     as.Date, as.Date.numeric
\end{verbatim}

\begin{Shaded}
\begin{Highlighting}[]
\FunctionTok{library}\NormalTok{(modelsummary)}
\end{Highlighting}
\end{Shaded}

\hypertarget{pregunta-de-investigaciuxf3n}{%
\section{Pregunta de investigación}\label{pregunta-de-investigaciuxf3n}}

\emph{¿En qué medida el manejo de políticas económicas para el
desarrollo afecta la fuga de mano de obra cualificada?}

\hypertarget{descripciuxf3n-de-los-modelos-a-estimar}{%
\section{Descripción de los modelos a
estimar}\label{descripciuxf3n-de-los-modelos-a-estimar}}

En el trabajo se usará el método de los Mínimos Cuadrados Ordinarios,
este consiste en cuatro regresiones, con el fin de saber el impacto que
tienen cada una de las variables entre hombres y mujeres en Estados
Unidos y Noruega respectivamente. Cada regresión cuenta con tres
variables independientes: Índice de Desarrollo Humano, Índice de
Percepción de Corrupción y Desempleo.

Detalladamente, por parte de Estados Unidos en la primera regresión se
quiere ver el impacto de las variables independientes anteriormente
mencionadas con la migración de hombres con alto capital humano hacía
los Estados Unidos, en la segunda el impacto de las variables
independientes con la migración de mujeres con alto capital humano hacía
los Estados Unidos. Con Noruega se busca de igual manera saber el
impacto de las variables independientes con diferenciación del género.

\hypertarget{resultados-de-los-modelos-estimados}{%
\subsection{Resultados de los modelos
estimados}\label{resultados-de-los-modelos-estimados}}

\begin{Shaded}
\begin{Highlighting}[]
\NormalTok{reg1}\OtherTok{\textless{}{-}} \FunctionTok{lm}\NormalTok{(HighMenNor }\SpecialCharTok{\textasciitilde{}}\NormalTok{ Development }\SpecialCharTok{+}\NormalTok{ Corruption }\SpecialCharTok{+}\NormalTok{ Unemployment, }\AttributeTok{data =}\NormalTok{ Base)}
\FunctionTok{summary}\NormalTok{(reg1)}
\end{Highlighting}
\end{Shaded}

\begin{verbatim}
## 
## Call:
## lm(formula = HighMenNor ~ Development + Corruption + Unemployment, 
##     data = Base)
## 
## Residuals:
##     Min      1Q  Median      3Q     Max 
## -310.95 -197.86  -96.74  -12.09 2384.68 
## 
## Coefficients:
##              Estimate Std. Error t value Pr(>|t|)
## (Intercept)   -80.438    230.748  -0.349    0.729
## Development   773.735    480.387   1.611    0.112
## Corruption    -56.900     42.228  -1.347    0.183
## Unemployment   -2.396      9.644  -0.248    0.805
## 
## Residual standard error: 416.9 on 61 degrees of freedom
##   (131 observations deleted due to missingness)
## Multiple R-squared:  0.04287,    Adjusted R-squared:  -0.004201 
## F-statistic: 0.9108 on 3 and 61 DF,  p-value: 0.4412
\end{verbatim}

\begin{Shaded}
\begin{Highlighting}[]
\NormalTok{reg2}\OtherTok{\textless{}{-}} \FunctionTok{lm}\NormalTok{(HighWomNor }\SpecialCharTok{\textasciitilde{}}\NormalTok{ Development }\SpecialCharTok{+}\NormalTok{ Corruption }\SpecialCharTok{+}\NormalTok{ Unemployment, }\AttributeTok{data =}\NormalTok{ Base)}
\FunctionTok{summary}\NormalTok{(reg2)}
\end{Highlighting}
\end{Shaded}

\begin{verbatim}
## 
## Call:
## lm(formula = HighWomNor ~ Development + Corruption + Unemployment, 
##     data = Base)
## 
## Residuals:
##     Min      1Q  Median      3Q     Max 
## -340.19 -198.79 -101.77  -13.01 2009.62 
## 
## Coefficients:
##              Estimate Std. Error t value Pr(>|t|)  
## (Intercept)   -85.381    234.596  -0.364   0.7172  
## Development   864.401    488.398   1.770   0.0817 .
## Corruption    -60.632     42.932  -1.412   0.1630  
## Unemployment   -4.083      9.804  -0.416   0.6786  
## ---
## Signif. codes:  0 '***' 0.001 '**' 0.01 '*' 0.05 '.' 0.1 ' ' 1
## 
## Residual standard error: 423.9 on 61 degrees of freedom
##   (131 observations deleted due to missingness)
## Multiple R-squared:  0.0513, Adjusted R-squared:  0.004644 
## F-statistic:   1.1 on 3 and 61 DF,  p-value: 0.3563
\end{verbatim}

\begin{Shaded}
\begin{Highlighting}[]
\NormalTok{reg3}\OtherTok{\textless{}{-}} \FunctionTok{lm}\NormalTok{(HighWomUS }\SpecialCharTok{\textasciitilde{}}\NormalTok{ Development }\SpecialCharTok{+}\NormalTok{ Corruption }\SpecialCharTok{+}\NormalTok{ Unemployment, }\AttributeTok{data =}\NormalTok{ Base)}
\FunctionTok{summary}\NormalTok{(reg3)}
\end{Highlighting}
\end{Shaded}

\begin{verbatim}
## 
## Call:
## lm(formula = HighWomUS ~ Development + Corruption + Unemployment, 
##     data = Base)
## 
## Residuals:
##    Min     1Q Median     3Q    Max 
## -60063 -29602 -19528  -1969 498053 
## 
## Coefficients:
##              Estimate Std. Error t value Pr(>|t|)  
## (Intercept)    -24239      45391  -0.534   0.5953  
## Development    164225      94498   1.738   0.0873 .
## Corruption     -10732       8307  -1.292   0.2013  
## Unemployment     -754       1897  -0.397   0.6924  
## ---
## Signif. codes:  0 '***' 0.001 '**' 0.01 '*' 0.05 '.' 0.1 ' ' 1
## 
## Residual standard error: 82010 on 61 degrees of freedom
##   (131 observations deleted due to missingness)
## Multiple R-squared:  0.04846,    Adjusted R-squared:  0.001663 
## F-statistic: 1.036 on 3 and 61 DF,  p-value: 0.3833
\end{verbatim}

\begin{Shaded}
\begin{Highlighting}[]
\NormalTok{reg4 }\OtherTok{\textless{}{-}} \FunctionTok{lm}\NormalTok{(HighMenUS }\SpecialCharTok{\textasciitilde{}}\NormalTok{ Development }\SpecialCharTok{+}\NormalTok{ Corruption }\SpecialCharTok{+}\NormalTok{ Unemployment, }\AttributeTok{data =}\NormalTok{ Base)}
\FunctionTok{summary}\NormalTok{(reg4)}
\end{Highlighting}
\end{Shaded}

\begin{verbatim}
## 
## Call:
## lm(formula = HighMenUS ~ Development + Corruption + Unemployment, 
##     data = Base)
## 
## Residuals:
##    Min     1Q Median     3Q    Max 
## -57886 -29653 -20325  -5272 632896 
## 
## Coefficients:
##              Estimate Std. Error t value Pr(>|t|)
## (Intercept)     -7404      51856  -0.143    0.887
## Development    132458     107958   1.227    0.225
## Corruption      -8727       9490  -0.920    0.361
## Unemployment    -1202       2167  -0.555    0.581
## 
## Residual standard error: 93690 on 61 degrees of freedom
##   (131 observations deleted due to missingness)
## Multiple R-squared:  0.02762,    Adjusted R-squared:  -0.02021 
## F-statistic: 0.5775 on 3 and 61 DF,  p-value: 0.632
\end{verbatim}

\begin{Shaded}
\begin{Highlighting}[]
\NormalTok{modelos}\OtherTok{\textless{}{-}} \FunctionTok{list}\NormalTok{(}\StringTok{"Modelo 1"} \OtherTok{=}\NormalTok{ reg1, }\StringTok{"Modelo 2"} \OtherTok{=}\NormalTok{ reg2, }\StringTok{"Modelo 3"} \OtherTok{=}\NormalTok{ reg3, }\StringTok{"Modelo 4"} \OtherTok{=}\NormalTok{ reg4)}
 
\NormalTok{resumen }\OtherTok{\textless{}{-}} \FunctionTok{stargazer}\NormalTok{(modelos, }\AttributeTok{type =} \StringTok{"text"}\NormalTok{, }
          \AttributeTok{out =} \StringTok{"tabla1.tex"}\NormalTok{, }
          \AttributeTok{title =} \StringTok{"Tabla 1 {-} Resultados de los modelos estimados"}\NormalTok{,}
          \AttributeTok{digits =} \DecValTok{1}\NormalTok{)}
\end{Highlighting}
\end{Shaded}

\begin{verbatim}
## 
## Tabla 1 - Resultados de los modelos estimados
## ==========================================================================
##                                           Dependent variable:             
##                               --------------------------------------------
##                               HighMenNor HighWomNor HighWomUS   HighMenUS 
##                                  (1)        (2)        (3)         (4)    
## --------------------------------------------------------------------------
## Development                     773.7      864.4*   164,224.7*  132,458.4 
##                                (480.4)    (488.4)   (94,498.4) (107,957.8)
##                                                                           
## Corruption                      -56.9      -60.6    -10,731.8   -8,726.7  
##                                 (42.2)     (42.9)   (8,306.8)   (9,489.9) 
##                                                                           
## Unemployment                     -2.4       -4.1      -754.0    -1,202.0  
##                                 (9.6)      (9.8)    (1,897.0)   (2,167.2) 
##                                                                           
## Constant                        -80.4      -85.4    -24,238.7   -7,404.5  
##                                (230.7)    (234.6)   (45,391.1) (51,856.1) 
##                                                                           
## --------------------------------------------------------------------------
## Observations                      65         65         65         65     
## R2                               0.04       0.1        0.05       0.03    
## Adjusted R2                     -0.004     0.005      0.002       -0.02   
## Residual Std. Error (df = 61)   416.9      423.9     82,010.4   93,691.1  
## F Statistic (df = 3; 61)         0.9        1.1        1.0         0.6    
## ==========================================================================
## Note:                                          *p<0.1; **p<0.05; ***p<0.01
\end{verbatim}

\hypertarget{gruxe1ficas-que-acompauxf1en-los-resultados}{%
\section{Gráficas que acompañen los
resultados}\label{gruxe1ficas-que-acompauxf1en-los-resultados}}

\begin{Shaded}
\begin{Highlighting}[]
\FunctionTok{options}\NormalTok{(}\AttributeTok{scipen =} \DecValTok{999}\NormalTok{)}
 
\FunctionTok{modelplot}\NormalTok{(modelos) }\SpecialCharTok{+} \FunctionTok{labs}\NormalTok{(}\AttributeTok{title =} \StringTok{"Resultados de los modelos"}\NormalTok{) }\SpecialCharTok{+}
  \FunctionTok{theme\_light}\NormalTok{()}\SpecialCharTok{+}
  \FunctionTok{theme}\NormalTok{(}\AttributeTok{plot.title =} \FunctionTok{element\_text}\NormalTok{(}\AttributeTok{hjust =} \FloatTok{0.5}\NormalTok{, }\AttributeTok{face =} \StringTok{"bold"}\NormalTok{, }\AttributeTok{family =} \StringTok{"serif"}\NormalTok{, }\AttributeTok{size =} \DecValTok{14}\NormalTok{), }\AttributeTok{axis.title.x =} \FunctionTok{element\_text}\NormalTok{(}\AttributeTok{face =} \StringTok{"bold.italic"}\NormalTok{, }\AttributeTok{family =} \StringTok{"serif"}\NormalTok{, }\AttributeTok{size =} \DecValTok{11}\NormalTok{), }\AttributeTok{axis.text =} \FunctionTok{element\_text}\NormalTok{(}\AttributeTok{face =} \StringTok{"italic"}\NormalTok{, }\AttributeTok{family =} \StringTok{"serif"}\NormalTok{, }\AttributeTok{size =} \DecValTok{11}\NormalTok{), }\AttributeTok{legend.text =} \FunctionTok{element\_text}\NormalTok{(}\AttributeTok{face =} \StringTok{"italic"}\NormalTok{, }\AttributeTok{family =} \StringTok{"serif"}\NormalTok{, }\AttributeTok{size =} \DecValTok{11}\NormalTok{), }\AttributeTok{legend.title =} \FunctionTok{element\_text}\NormalTok{(}\AttributeTok{size =} \DecValTok{11}\NormalTok{, }\AttributeTok{face =} \StringTok{"bold"}\NormalTok{, }\AttributeTok{family =} \StringTok{"serif"}\NormalTok{))}\SpecialCharTok{+}
  \FunctionTok{scale\_color\_brewer}\NormalTok{(}\AttributeTok{palette =} \StringTok{"Set3"}\NormalTok{)}
\end{Highlighting}
\end{Shaded}

\includegraphics{Análisis-Preliminar_files/figure-latex/unnamed-chunk-3-1.pdf}

\hypertarget{revisiuxf3n-de-los-principales-supuestos-de-los-modelos-estimados}{%
\section{Revisión de los principales supuestos de los modelos
estimados}\label{revisiuxf3n-de-los-principales-supuestos-de-los-modelos-estimados}}

\begin{Shaded}
\begin{Highlighting}[]
\NormalTok{res1 }\OtherTok{\textless{}{-}}\NormalTok{ reg1}\SpecialCharTok{$}\NormalTok{residuals}

\NormalTok{res2 }\OtherTok{\textless{}{-}}\NormalTok{ reg2}\SpecialCharTok{$}\NormalTok{residuals}

\NormalTok{res3 }\OtherTok{\textless{}{-}}\NormalTok{ reg3}\SpecialCharTok{$}\NormalTok{residuals}

\NormalTok{res4 }\OtherTok{\textless{}{-}}\NormalTok{ reg4}\SpecialCharTok{$}\NormalTok{residuals}
\end{Highlighting}
\end{Shaded}

\hypertarget{test-de-normalidad}{%
\subsubsection{Test de normalidad}\label{test-de-normalidad}}

Test Jarque-Bera

\begin{Shaded}
\begin{Highlighting}[]
\FunctionTok{jarque.bera.test}\NormalTok{(res1)}
\end{Highlighting}
\end{Shaded}

\begin{verbatim}
## 
##  Jarque Bera Test
## 
## data:  res1
## X-squared = 980.26, df = 2, p-value < 0.00000000000000022
\end{verbatim}

\begin{Shaded}
\begin{Highlighting}[]
\FunctionTok{jarque.bera.test}\NormalTok{(res2)}
\end{Highlighting}
\end{Shaded}

\begin{verbatim}
## 
##  Jarque Bera Test
## 
## data:  res2
## X-squared = 316.42, df = 2, p-value < 0.00000000000000022
\end{verbatim}

\begin{Shaded}
\begin{Highlighting}[]
\FunctionTok{jarque.bera.test}\NormalTok{(res3)}
\end{Highlighting}
\end{Shaded}

\begin{verbatim}
## 
##  Jarque Bera Test
## 
## data:  res3
## X-squared = 1728.3, df = 2, p-value < 0.00000000000000022
\end{verbatim}

\begin{Shaded}
\begin{Highlighting}[]
\FunctionTok{jarque.bera.test}\NormalTok{(res4)}
\end{Highlighting}
\end{Shaded}

\begin{verbatim}
## 
##  Jarque Bera Test
## 
## data:  res4
## X-squared = 3555.4, df = 2, p-value < 0.00000000000000022
\end{verbatim}

\hypertarget{homocedasticidad}{%
\subsubsection{Homocedasticidad}\label{homocedasticidad}}

Brush Pagan

\begin{Shaded}
\begin{Highlighting}[]
\FunctionTok{bptest}\NormalTok{(reg1)}
\end{Highlighting}
\end{Shaded}

\begin{verbatim}
## 
##  studentized Breusch-Pagan test
## 
## data:  reg1
## BP = 3.5931, df = 3, p-value = 0.3089
\end{verbatim}

\begin{Shaded}
\begin{Highlighting}[]
\FunctionTok{bptest}\NormalTok{(reg2)}
\end{Highlighting}
\end{Shaded}

\begin{verbatim}
## 
##  studentized Breusch-Pagan test
## 
## data:  reg2
## BP = 3.7709, df = 3, p-value = 0.2873
\end{verbatim}

\begin{Shaded}
\begin{Highlighting}[]
\FunctionTok{bptest}\NormalTok{(reg3)}
\end{Highlighting}
\end{Shaded}

\begin{verbatim}
## 
##  studentized Breusch-Pagan test
## 
## data:  reg3
## BP = 0.26793, df = 3, p-value = 0.9659
\end{verbatim}

\begin{Shaded}
\begin{Highlighting}[]
\FunctionTok{bptest}\NormalTok{(reg4)}
\end{Highlighting}
\end{Shaded}

\begin{verbatim}
## 
##  studentized Breusch-Pagan test
## 
## data:  reg4
## BP = 0.17166, df = 3, p-value = 0.982
\end{verbatim}

\hypertarget{no-autocorrelaciuxf3n}{%
\subsubsection{No autocorrelación}\label{no-autocorrelaciuxf3n}}

Prueba Durbin Watson

\begin{Shaded}
\begin{Highlighting}[]
\FunctionTok{dwtest}\NormalTok{(reg1)}
\end{Highlighting}
\end{Shaded}

\begin{verbatim}
## 
##  Durbin-Watson test
## 
## data:  reg1
## DW = 2.0449, p-value = 0.5779
## alternative hypothesis: true autocorrelation is greater than 0
\end{verbatim}

\begin{Shaded}
\begin{Highlighting}[]
\FunctionTok{dwtest}\NormalTok{(reg2)}
\end{Highlighting}
\end{Shaded}

\begin{verbatim}
## 
##  Durbin-Watson test
## 
## data:  reg2
## DW = 1.8878, p-value = 0.3293
## alternative hypothesis: true autocorrelation is greater than 0
\end{verbatim}

\begin{Shaded}
\begin{Highlighting}[]
\FunctionTok{dwtest}\NormalTok{(reg3)}
\end{Highlighting}
\end{Shaded}

\begin{verbatim}
## 
##  Durbin-Watson test
## 
## data:  reg3
## DW = 1.8739, p-value = 0.309
## alternative hypothesis: true autocorrelation is greater than 0
\end{verbatim}

\begin{Shaded}
\begin{Highlighting}[]
\FunctionTok{dwtest}\NormalTok{(reg4)}
\end{Highlighting}
\end{Shaded}

\begin{verbatim}
## 
##  Durbin-Watson test
## 
## data:  reg4
## DW = 1.9782, p-value = 0.4693
## alternative hypothesis: true autocorrelation is greater than 0
\end{verbatim}

\hypertarget{dos-puxe1rrafos-con-un-anuxe1lisis-preliminar-de-los-resultados}{%
\section{Dos párrafos con un análisis preliminar de los
resultados}\label{dos-puxe1rrafos-con-un-anuxe1lisis-preliminar-de-los-resultados}}

\end{document}
